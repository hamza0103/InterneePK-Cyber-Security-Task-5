\documentclass[12pt,a4paper]{article}

% --- PACKAGES ---
\usepackage[utf8]{inputenc}
\usepackage[T1]{fontenc}
\usepackage{helvet}
\renewcommand{\familydefault}{\sfdefault} 
\usepackage{geometry}
\usepackage[table]{xcolor} % <--- THIS FIXED THE ERROR
\usepackage{setspace}
\usepackage{tcolorbox}
\usepackage{fancyhdr}

% --- CONFIGURATION ---
\geometry{left=2.5cm, right=2.5cm, top=3cm, bottom=3cm}
\definecolor{primary}{RGB}{0, 51, 102}    % Navy Blue
\definecolor{graytext}{RGB}{80, 80, 80}   % Dark Gray
\begin{document}

% =========================================================
% PAGE 1: CENTERED PROFESSIONAL TITLE PAGE
% =========================================================

\begin{titlepage}
    \centering
    \onehalfspacing
    
    % --- Top Branding ---
    \vspace*{0.5cm}
    {\Huge \bfseries \textcolor{primary}{INTERNEE.PK}} \\[0.2cm]
    {\Large \textcolor{graytext}{Cyber Security Internship Program}} \\[0.2cm]
    {\small \textbf{BATCH 2026}}
    
    \vspace{3.5cm}
    
    % --- The Title Block (The "Power" Section) ---
    % We use thick lines to frame the title
    \hrule height 2pt \vspace{0.8cm}
    
    {\fontsize{30}{36}\selectfont \textbf{INCIDENT RESPONSE PLAN}} \\[0.5cm]
    {\Large \textit{Ransomware Simulation \& Defense Protocols}}
    
    \vspace{0.8cm} \hrule height 2pt
    
    \vspace{3cm}
    
    % --- Document Details (Grid Layout) ---
    \begin{minipage}{0.45\textwidth}
        \begin{flushleft}
            \textbf{\textcolor{primary}{PREPARED BY:}} \\
            Muhammad Hamza Hayat \\
            Cyber Security Intern
        \end{flushleft}
    \end{minipage}
    \hfill
    \begin{minipage}{0.45\textwidth}
        \begin{flushright}
            \textbf{\textcolor{primary}{SUBMITTED TO:}} \\
            Internee.pk Management \\
            Security Operations Center (SOC)
        \end{flushright}
    \end{minipage}
    
    \vfill
    
    % --- Date & Version ---
    \textbf{Date:} \today \\
    \textbf{Version:} 1.0 
    
    \vspace{1.5cm}
    
    % --- Bottom Footer Box ---
    \fcolorbox{primary}{white}{
        \begin{minipage}{0.85\textwidth}
        \centering \vspace{0.3cm}
        \small \textbf{COMPLIANCE STATEMENT} \\
his document adheres to \textbf{NIST SP 800-61 Rev. 2} standards for Computer Security Incident Handling.
        \vspace{0.3cm}
        \end{minipage}
    }

\end{titlepage}

% =========================================================
% PAGE 2: GOVERNANCE & ROLES (FIXED SPACING)
% =========================================================

\newpage
\newgeometry{left=2.5cm, right=2.5cm, top=2.5cm, bottom=2.5cm}

% --- Section 1: Introduction ---
\section*{1.0 Executive Summary}
The purpose of this \textbf{Cyber Incident Response Plan (CIRP)} is to provide a structured framework for handling cybersecurity incidents at \textbf{Internee.pk}. By defining clear roles and pre-approved procedures, we aim to reduce the \textit{Mean Time to Respond (MTTR)} and minimize the impact of security incidents on business operations.

% --- Section 2: Scope ---
\section*{2.0 Scope of Applicability}
This plan applies to all systems, networks, and data owned or operated by the organization.
\begin{itemize}
    \item \textbf{Infrastructure:} Workstations, Servers (Windows/Linux), and Network Devices.
    \item \textbf{Cloud Assets:} AWS Instances, S3 Buckets, and Hosted Databases.
    \item \textbf{Users:} All employees, contractors, and interns with network access.
\end{itemize}

% --- Section 3: The Incident Response Team (IRT) ---
\section*{3.0 Incident Response Team (IRT)}
A clearly defined chain of command is critical during a crisis. The following roles constitute the authorized IRT.



\begin{center}
\renewcommand{\arraystretch}{1.6}
% Using \raggedright and \arraybackslash to fix the \hbox stretching errors
\begin{tabular}{|p{0.32\textwidth}|>{\raggedright\arraybackslash}p{0.58\textwidth}|}
\hline
\rowcolor{primary} \textbf{\textcolor{white}{ROLE}} & \textbf{\textcolor{white}{RESPONSIBILITIES}} \\ \hline

\textbf{Incident Commander} & 
Highest authority. Coordinates response, approves containment strategies, and manages stakeholder communication. \\ \hline

\textbf{Lead Security Analyst} & 
Technical Lead. Investigates logs (SIEM/Wazuh), identifies root causes, and determines breach scope. \\ \hline

\textbf{IT Operations Lead} & 
Execution. Performs system isolation, password resets, firewall blocking, and backup restoration. \\ \hline

\end{tabular}
\end{center}

\vfill
\begin{tcolorbox}[colback=white, colframe=graytext, title=\textbf{Authority to Act}]
\small The \textbf{Incident Commander} is authorized to sever network connections to the internet without prior approval if a critical threat is detected spreading laterally.
\end{tcolorbox}

\newpage

% =========================================================
% PAGE 3: SEVERITY MATRIX & DETECTION (ERROR-FREE)
% =========================================================

\newpage
\newgeometry{left=2.5cm, right=2.5cm, top=2.5cm, bottom=2.5cm}

% --- Section 4: Incident Severity Matrix ---
\section*{4.0 Incident Severity Matrix}
To ensure efficient resource allocation, all incidents must be classified upon detection. This matrix determines the urgency and escalation path.

\begin{center}
\renewcommand{\arraystretch}{1.8}
% We use \raggedright to prevent the underfull \hbox error in narrow columns
\begin{tabular}{|p{0.18\textwidth}|>{\raggedright\arraybackslash}p{0.48\textwidth}|p{0.24\textwidth}|}
\hline
\rowcolor{primary} \textbf{\textcolor{white}{SEVERITY}} & \textbf{\textcolor{white}{CRITERIA}} & \textbf{\textcolor{white}{RESPONSE}} \\ \hline

\textbf{\textcolor{red}{CRITICAL}} & 
Widespread Ransomware, Data Breach of PII, or Total Service Outage. & Immediate (24/7) \\ \hline

\textbf{\textcolor{orange}{HIGH}} & 
Targeted malware on a server or unauthorized admin access. & Within 1 Hour \\ \hline

\textbf{\textcolor{blue}{MEDIUM}} & 
Isolated virus on a workstation or suspicious login attempts. & Within 4 Hours \\ \hline

\textbf{LOW} & 
Spam, Adware, or minor policy violations. & Next Business Day \\ \hline

\end{tabular}
\end{center}

\vspace{0.8cm}

% --- Section 5: Detection & Analysis ---
\section*{5.0 Detection \& Analysis Procedures}
The goal of this phase is to confirm whether a security event is a true incident or a false positive.

\subsection*{5.1 Detection Sources}
Internee.pk utilizes the following sources for initial detection:
\begin{itemize}
    \item \textbf{SIEM Alerts:} Monitoring Wazuh agents for file integrity and log anomalies.
    \item \textbf{Endpoint Security:} Antivirus and EDR alerts on workstations.
    \item \textbf{User Reporting:} Employees reporting suspicious emails or system behavior.
\end{itemize}

\subsection*{5.2 Verification Steps}
Upon receiving an alert, the Lead Security Analyst must:
\begin{enumerate}
    \item \textbf{Analyze:} Review timestamps, source IPs, and user accounts involved.
    \item \textbf{Correlate:} Check if other systems are showing similar anomalies.
    \item \textbf{Validate:} Determine if the activity is authorized (e.g., scheduled maintenance).
\end{enumerate}

\vfill
\begin{tcolorbox}[colback=primary!5, colframe=primary, title=\textbf{Pro Tip}]
\small Always assume a "Critical" severity for any alert involving multiple encrypted files until proven otherwise.
\end{tcolorbox}

\newpage

% =========================================================
% PAGE 4: RANSOMWARE SIMULATION PLAYBOOK (FIXED)
% =========================================================

\newpage
\newgeometry{left=2.5cm, right=2.5cm, top=2.5cm, bottom=2.5cm}

\section*{6.0 Ransomware Simulation Playbook}
This section simulates a response to a high-impact ransomware attack. This playbook is designed to be executed immediately upon the detection of encryption artifacts.

\subsection*{6.1 Scenario Overview}
\begin{itemize}
    \item \textbf{Trigger:} A Wazuh File Integrity Monitor (FIM) alert detects rapid file extension changes (.crypt) on the Finance Server.
    \item \textbf{Initial Assessment:} 25\% of files are encrypted; a ransom note is present on the desktop.
\end{itemize}

\subsection*{6.2 Phase 1: Containment (Stop the Spread)}
\begin{enumerate}
    \item \textbf{Network Isolation:} The IT Operations Lead must immediately disable the switch port or disconnect the virtual NIC of the infected server.
    \item \textbf{Lateral Movement Check:} Block all SMB (Port 445) and RDP (Port 3389) traffic internally to prevent the ransomware from spreading.
    \item \textbf{Endpoint Lockdown:} Force a global password reset for all Domain Admin accounts to prevent credential harvesting.
\end{enumerate}

\subsection*{6.3 Phase 2: Eradication \& Investigation}
\begin{itemize}
    \item \textbf{Root Cause Analysis:} Use Wazuh logs to identify the "Patient Zero" (the first machine infected).
    \item \textbf{Malware Removal:} Perform a full disk scan using offline tools. If the infection is deep, the system must be wiped and re-installed from a "Golden Image."
\end{itemize}

\subsection*{6.4 Phase 3: Recovery \& Restoration}
% --- Fixed Box Title to avoid Overfull \hbox ---
\begin{tcolorbox}[colback=red!5, colframe=red, title=\textbf{POLICY: RANSOM NON-PAYMENT}]
\small Internee.pk maintains a strict policy against paying ransoms. This prevents funding criminal activity and does not guarantee data recovery.
\end{tcolorbox}

\begin{flushleft} % This kills the \hbox error for the list
\begin{enumerate}
    \item \textbf{Backup Validation:} Verify the integrity of the most recent offline backup (Veeam/AWS Snapshots).
    \item \textbf{Incremental Restore:} Restore data to a "Clean Room" environment first to ensure no malware is hidden.
    \item \textbf{Production Re-entry:} Gradually bring services back online while keeping monitoring levels at "Maximum" for 72 hours.
\end{enumerate}
\end{flushleft}

\vfill
\begin{center}
    \small Note: All actions taken during this simulation must be logged in the Incident Chronology for post-mortem analysis.
\end{center}


\newpage

% =========================================================
% PAGE 5: STAFF TRAINING & EMERGENCY CONTACTS
% =========================================================

\newpage
\newgeometry{left=2.5cm, right=2.5cm, top=2.5cm, bottom=2.5cm}

\section*{7.0 Staff Training \& Awareness}
Human error is the leading cause of initial compromise. To support this IRP, \textbf{Internee.pk} shall conduct the following training modules for all staff.

\subsection*{7.1 Phishing Simulation}
\begin{itemize}
    \item \textbf{Frequency:} Quarterly unannounced simulations.
    \item \textbf{Objective:} Train users to identify suspicious senders, mismatched URLs, and urgent/threatening language.
    \item \textbf{Reporting:} Users are instructed to use the "Report Phishing" button rather than deleting the email.
\end{itemize}

\subsection*{7.2 Incident Reporting Protocol}
In the event of suspicious system behavior (slow performance, disappearing files, or pop-ups), staff must:
\begin{enumerate}
    \item \textbf{Stop:} Cease all activity on the device.
    \item \textbf{Report:} Call the IT Helpdesk immediately.
    \item \textbf{Do Not Shutdown:} Leave the machine on but disconnect the Ethernet cable (to preserve RAM artifacts for forensics).
\end{enumerate}

\vspace{1cm}

\section*{8.0 Emergency Contact List}
In a "Critical" severity incident, the following contacts are to be used for immediate escalation.

\begin{center}
\renewcommand{\arraystretch}{1.8}
\begin{tabular}{|p{0.3\textwidth}|p{0.3\textwidth}|p{0.3\textwidth}|}
\hline
\rowcolor{primary} \textbf{\textcolor{white}{DEPARTMENT}} & \textbf{\textcolor{white}{PRIMARY CONTACT}} & \textbf{\textcolor{white}{PHONE / EXT}} \\ \hline

\textbf{IT Security (SOC)} & Lead Security Analyst & +92-XXX-XXXXXXX \\ \hline

\textbf{Management} & Incident Commander & +92-XXX-XXXXXXX \\ \hline

\textbf{Cloud Provider} & AWS Support (24/7) & 1-800-AWS-HELP \\ \hline

\textbf{Law Enforcement} & FIA Cybercrime Wing & 1991 \\ \hline

\end{tabular}
\end{center}

\vfill
\begin{center}
    \hrule height 1pt \vspace{0.3cm}
    \textbf{END OF DOCUMENT} \\
    \small \textit{This plan shall be reviewed and updated annually or after any major security incident.}
\end{center}


% =========================================================
% PAGE 6: POST-INCIDENT REPORT (PIR) TEMPLATE
% =========================================================

\newpage
\newgeometry{left=2.5cm, right=2.5cm, top=2.5cm, bottom=2.5cm}

\section*{9.0 Post-Incident Report Template}
This document must be completed within 72 hours of incident resolution. The goal is to identify root causes and improve future response capabilities.

\subsection*{9.1 Incident Summary}
\begin{tcolorbox}[colback=white, colframe=black, height=4cm]
\small \textit{Describe the timeline, the systems affected, and the initial point of entry (e.g., Phishing, Unpatched Vulnerability).}
\end{tcolorbox}
\subsection*{9.2 Resolution \& Recovery Actions}
\begin{flushleft}
\begin{itemize}
    \item \textbf{Containment Method:} \rule{8cm}{0.4pt}
    \item \textbf{Backup Success:} [ ] Yes [ ] No (If no, why?)
    \item \textbf{Data Loss Status:} \rule{8cm}{0.4pt}
\end{itemize}
\end{flushleft}

\subsection*{9.3 Lessons Learned \& Improvements}
\begin{center}
\renewcommand{\arraystretch}{2}
\begin{tabular}{|p{0.45\textwidth}|p{0.45\textwidth}|}
\hline
\rowcolor{lightgray} \textbf{What went well?} & \textbf{What could be improved?} \\ \hline
\vspace{1.5cm} & \vspace{1.5cm} \\ \hline
\end{tabular}
\end{center}

\subsection*{9.4 Corrective Action Plan}
\begin{flushleft}
\begin{enumerate}
    \item \textbf{Immediate Task:} \rule{8cm}{0.4pt}
    \item \textbf{Security Patching:} \rule{8cm}{0.4pt}
    \item \textbf{Staff Retraining Required:} [ ] Yes [ ] No
\end{enumerate}
\end{flushleft}

\vfill
\begin{flushright}
    \begin{minipage}{0.4\textwidth}
        \centering
        \hrule
        \vspace{0.2cm}
        \textbf{Incident Commander Signature} \\
        \small Date: \rule{3cm}{0.4pt}
    \end{minipage}
\end{flushright}

\newpage
\section*{References \& Frameworks}
\begin{flushleft} % Prevents the \hbox stretching error
\begin{itemize}
    \item \textbf{MITRE ATT\&CK:} Techniques T1486 (Data Encrypted for Impact) and T1566 (Phishing) were used to model the Ransomware Simulation.
    \item \textbf{NIST SP 800-61 Rev. 2:} Followed for the Incident Handling Lifecycle standards.
    \item \textbf{Internee.pk Internship Tasks:} Aligned with Wazuh deployment and secure cloud infrastructure projects.
\end{itemize}
\end{flushleft}

      
\end{document}
